% !TeX program = pdfLaTeX
\documentclass[smallextended]{svjour3}       % onecolumn (second format)
%\documentclass[twocolumn]{svjour3}          % twocolumn
%
\smartqed  % flush right qed marks, e.g. at end of proof
%
\usepackage{amsmath}
\usepackage{graphicx}
\usepackage[utf8]{inputenc}

\usepackage[hyphens]{url} % not crucial - just used below for the URL
\usepackage{hyperref}
\providecommand{\tightlist}{%
  \setlength{\itemsep}{0pt}\setlength{\parskip}{0pt}}

%
% \usepackage{mathptmx}      % use Times fonts if available on your TeX system
%
% insert here the call for the packages your document requires
%\usepackage{latexsym}
% etc.
%
% please place your own definitions here and don't use \def but
% \newcommand{}{}
%
% Insert the name of "your journal" with
% \journalname{myjournal}
%

%% load any required packages here




\usepackage{xcolor}
\usepackage{booktabs}
\usepackage{longtable}
\usepackage{array}
\usepackage{multirow}
\usepackage{wrapfig}
\usepackage{float}
\usepackage{colortbl}
\usepackage{pdflscape}
\usepackage{tabu}
\usepackage{threeparttable}
\usepackage{threeparttablex}
\usepackage[normalem]{ulem}
\usepackage{makecell}

\begin{document}

\title{Correlates of Cycling Flows in Hamilton, Ontario - Fastest, Quietest, or
Balanced Routes? }



\author{  Elise Desjardins \and  Christopher D. Higgins \and  Darren M. Scott \and  Emma Apatu \and  Antonio Páez \and  }


\institute{
        Elise Desjardins \at
     School of Geography and Earth Sciences, McMaster University \\
     \email{\href{mailto:desjae@mcmaster.ca}{\nolinkurl{desjae@mcmaster.ca}}}  %  \\
%             \emph{Present address:} of F. Author  %  if needed
    \and
        Christopher D. Higgins \at
     Department of Human Geography, University of Toronto \\
     \email{\href{mailto:cd.higgins@utoronto.ca}{\nolinkurl{cd.higgins@utoronto.ca}}}  %  \\
%             \emph{Present address:} of F. Author  %  if needed
    \and
        Darren M. Scott \at
     School of Geography and Earth Sciences, McMaster University \\
     \email{\href{mailto:scottdm@mcmaster.ca}{\nolinkurl{scottdm@mcmaster.ca}}}  %  \\
%             \emph{Present address:} of F. Author  %  if needed
    \and
        Emma Apatu \at
     Department of Health Research Methods, Evidence, and Impact, McMaster
 University \\
     \email{\href{mailto:apatue@mcmaster.ca}{\nolinkurl{apatue@mcmaster.ca}}}  %  \\
%             \emph{Present address:} of F. Author  %  if needed
    \and
        Antonio Páez \at
     School of Geography and Earth Sciences, McMaster University \\
     \email{\href{mailto:paezha@mcmaster.ca}{\nolinkurl{paezha@mcmaster.ca}}}  %  \\
%             \emph{Present address:} of F. Author  %  if needed
    \and
    }

\date{Received: date / Accepted: date}
% The correct dates will be entered by the editor


\maketitle

\begin{abstract}
Cycling is an increasingly popular mode of travel in Canadian urban
areas, like the Greater Toronto and Hamilton Area (GTHA). While trip
origins and destinations can be inferred from travel surveys, data on
route choice is often not collected which makes it challenging to
capture the attributes of routes travelled by cyclists. With new
algorithms for cycle routing it is now possible to infer routes. Using
bicycle trip records from the most recent regional travel survey, a
spatial interaction model is developed to investigate the built
environment correlates of cycling flows in Hamilton, Ontario, a
mid-sized city part of the GTHA. A feature of the analysis is the use of
CycleStreets to compare the distance and time according to different
routes inferred between trip zones of origin and destination. In
addition, network autocorrelation is accounted for in the estimated
models. The most parsimonious model suggests that shortest-path quietest
routes that minimize traffic best explain the pattern of travel by
bicycle in Hamilton. Commercial and office locations and points of
interest at the zone of origin negatively correlate with the production
of trips, while multiple land uses and the availability of jobs at the
zone of destination are trip attractors. The use of a route planner
offers a novel approach to modelling and understanding cycling flows
within a city. This may be useful for transportation planners to infer
different types of routes that cyclists may seek out and consider these
in travel demand models.
\\
\keywords{
        cycling \and
        spatial interaction modelling \and
        route choice \and
    }


\end{abstract}


\def\spacingset#1{\renewcommand{\baselinestretch}%
{#1}\small\normalsize} \spacingset{1}


\hypertarget{sec:intro}{%
\section{Introduction}\label{sec:intro}}

Cycling is an increasingly popular mode of travel in Canadian urban
areas. From 1996 to 2016 the number of people commuting to work by
bicycle in Canadian census metropolitan areas increased by 87.9\% and
the share of bicycle commute trips grew from 1.2\% to 1.6\% (Statistics
Canada 2017). Such modal shifts have been prompted in part by the widely
recognized health and environmental benefits associated with cycling.
Compared to other transportation modes, travelling by bicycle is more
enjoyable (Páez and Whalen 2010), and is associated with better
self-perceived health (Avila-Palencia et al. 2018) and reduced risk of
chronic disease (Celis-Morales, Lyall, and Welsh 2017; Oja et al. 2011).
Furthermore, cycling also leads to reduced greehouse gas emissions
(Zahabi et al. 2016) and improved air and noise pollution (De Nazelle et
al. 2011). These benefits serve as motivation for cities to encourage
more travel by this mode, but this requires effort to put cycling on par
with other modes of transportation at a policy level. For this reason,
many Canadian cities have integrated cycling in their transportation
plans in recent years (\emph{inter alia}, see City of Calgary 2011; City
of Montreal 2017; City of Vancouver 2012) and have implemented a range
of interventions and strategies that have been effective in increasing
cycling (Assunçao-Denis and Tomalty 2019; Verlinden, Y and Manaugh, K
and Savan, B and Smith Lea, N and Tomalty, R and Winters, M 2019).

The City of Hamilton, a mid-sized city located in the Greater Toronto
and Hamilton Area (GTHA) urban region approximately 50 km from Toronto,
has experienced an increase in cycling. Approximately one third of all
trips in this area are 5 km or less, which is widely considered to be a
bikeable distance. In 2016, 1.2\% of all trips in Hamilton were made by
bicycle according to the latest \emph{Transportation Tomorrow Survey},
the regional travel survey conducted every 5 years (Data Management
Group 2018). This was a two-fold increase from the 2011 survey results
when the cycling mode share was only 0.6\% (Data Management Group 2014).
Hamilton has also been identified as a city where cycling levels could
substantially increase to approximately 35\% of the mode share (Mitra et
al. 2016). The increase in bicycle trips in Hamilton between 2011 and
2016 occurred over the same period that cycling interventions were
implemented, such as new cycling facilities and a public bicycle-share
program. In other words, we suggest that Hamilton can be characterized
as a developing cycling city (see Liu et al. 2020) because efforts to
increase cycling have been implemented and cycling levels are growing.
Recent studies have used GPS data from the bicycle-share program to
conduct route choice analysis (Lu, Scott, and Dalumpines 2018) or
explore influences on bike share ridership (Scott and Ciuro 2019).
However, we still know relatively little about trips in this city beyond
those made by bikeshare. To date, there has been no published research
that has investigated the pattern of bicycle trips in Hamilton using
data from the 2016 \emph{Transportation Tomorrow Survey}. Our
understanding of the spatial distribution of such trips in a mid-sized
developing cycling city and the influence of the built environment is
also limited.

To address these gaps in knowledge, the objective of this study is to
investigate the correlates of cycling flows in Hamilton. This paper
describes the development of a spatial interaction model (a component of
the four-step travel model; see Dios Ortúzar and Willumsen 2011) to test
the level of cycling flows against various attributes at the zones of
origin and destination. Travel surveys are typically rich in terms of
information about where trips start or end, but are less informative
with respect to route characteristics, which often have to be inferred.
For this reason, a feature of the analysis is the use of an algorithm
for cycle routing, \emph{CycleStreets}, to infer and compare different
routes between the zones of origin and destination instead of using only
the shortest-path distance. This algorithm identifies routes according
to various attributes and characterizes them as \emph{fastest},
\emph{quietest}, and \emph{balanced} routes. The distance and time from
the zone of origin to destination along each inferred route serve as
measures of cost in the analysis. The following two questions are
addressed: 1) \emph{Which attributes at the zones of origin and
destination influence cycling trip flows in Hamilton?}; and 2)
\emph{Which type of route best explains the pattern of travel by bicycle
in Hamilton?} In addition, residuals from the spatial interaction model
are analyzed using a spatial autocorrelation statistic to assess the
model's compliance with the assumption of independently distributed
residuals. Future opportunities for research and practice, including
assessment of the built environment along select routes identified by
the algorithm, are also discussed.

Note that all data and code used in this research are available online.
The source for this paper is an R markdown document that can be obtained
from the following GitHub repository:

\begin{quote}
\url{https://github.com/paezha/Correlates-of-Cycling-Flows-Route-Types}
\end{quote}

\hypertarget{sec:background}{%
\section{Background}\label{sec:background}}

A diversity of factors influence the decision to commute by bicycle
ranging from the natural and built environments to individual and
household characteristics (Heinen, Wee, and Maat 2010). Where people
live, work, and play is particularly important because it influences the
transport modes available to them, the destinations and amenities that
they can access, and the routes they can travel to get from A to B. As
such, the built environment receives a lot of attention in cycling
research because factors that are known to influence cycling can be
modified by urban and transportation planners to potentially shift a
large number of currently motorized trips. Population-based travel
surveys are useful for understanding cycling activity and patterns at
the city level (Handy, Wee, and Kroesen 2014) which can, in turn,
support strategic investments where cycling levels have the potential to
increase.

The \emph{behavioral model of the environment} was proposed as a
theoretical framework for environmental audits to identify the
determinants of walking and cycling at three different scales that make
up any trip (Moudon and Lee 2003). According to this framework, all
three spatial areas (i.e., i) the characteristics of the origin, ii)
thedestination, and iii) the route) are important and necessary to
assess the influence of the built environment on walking and cycling.
These modes, more so than motorized travel, allow a traveller to
interact more intimately with the micro-level environment (Moniruzzaman
and Páez 2012, 2016; Moudon and Lee 2003). This type of framework holds
true for bicycle trip analysis as well. Winters et al.~(2010) conducted
a study measuring built environment variables at three different scales
in Vancouver, Canada and found that the built environment around the
origin and destination, as well as along the route, are indeed different
and influence cycling in different ways. This emphasizes the need for
travel behaviour models to capture environmental attributes along
different parts of the trip, not just at the zone of origin and
destination or the community-level.

\hypertarget{macro}{%
\subsection{Macro-Level Built Environment Factors}\label{macro}}

The urban form at the places where bicycle trips originate and end is
important (Scott and Ciuro 2019). This topic is well-documented in the
cycling literature. Land use mix, whereby people can reach a variety of
amenities within a distance that is comfortable to cycle, influences
travel by bicycle (Cervero, Denman, and Jin 2019; Sallis et al. 2013;
Winters et al. 2010; Zhao 2014). For instance, Heesch et al.~(2015)
found that shorter distances to destinations, including a business
district with jobs and a river where there are bicycle paths, increased
the odds of cycling in Brisbane, Australia. Higher densities of
population (Nielsen and Skov-Petersen 2018; Nordengen et al. 2019;
Schneider and Stefanich 2015; Winters et al. 2010) and employment {[}Le,
Buehler, and Hankey (2018); Zhao2014{]} are other important factors. In
the case of cities with low levels of cycling, access to bicycles can
make this mode more attractive. Cole-Hunter et al.~(2015) report that
public bicycle share stations near the residence was a significant
positive determinant of commuting by bicycle. The quality of the urban
environment also matters. Areas with trees and green space also
associated with more cycling (Cole-Hunter et al. 2015; Le, Buehler, and
Hankey 2018; Mertens et al. 2017).

In most studies, a combination of these attributes are found to
influence cycling, which suggests that multiple factors are needed to
create spaces that ultimately encourage people to cycle (Cervero,
Denman, and Jin 2019). Higher levels of cycling are typically observed
in neighbourhoods with good street connectivity, supportive
infrastructure, and a variety of amenities that can be reached in a
short distance. However, there is variation in the relative influence of
these attributes across studies and across places, which might reveal
different effects that are related to contextual behaviours, or planning
and transportation policies. For example, residential density (Scott and
Ciuro 2019; Zhao 2014) and the presence of cycling infrastructure
(Moudon et al. 2005) are not always a significant factors. Therefore,
additional analysis to determine the influence of specific attributes on
cycling levels is important in developing cycling cities, where such
studies have not been previously conducted, which can inform new
strategies to induce the uptake of cycling.

\hypertarget{route}{%
\subsection{Micro-Level Route Factors}\label{route}}

Cycling infrastructure is often identified as an important attribute in
bicycle-friendly cities. It is thought to be fundamental for encouraging
more bicycle trips in cities that are predominantly car-centric (Adam,
Jones, and Brömmelstroet 2020). The provision of, or proximity to,
infrastructure has been found to have a influence or assosication with
cycling behaviour (\emph{inter alia}, see Buehler and Pucher 2012;
Buehler and Dill 2016; Dill and Carr 2003; Mertens et al. 2017; Winters
et al. 2010). For example, Le et al.~(2018) found that cycling
facilities had a strong association with bicycle volume and traffic
based on their analysis of 20 metropolitan statistical areas in the
United States. Infrastructure can be very influential - a new bicycle
lane in Oslo, Norway attracted trips by shifting cyclists from other
parallel routes (Pritchard, Bucher, and Frøyen 2019). This suggests that
it is not uncommon for preferred routes to change as new facilities are
built over time and they are incorporated into daily trips. Furthermore,
infrastructure can also increase perceptions of cycling safety
(Branion-Calles et al. 2019) which may encourage more trips. At the very
least, cycling infrastructure is a visual and physical sign that streets
can accommodate people who choose to travel using this mode.

Studies examining the characteristics along routes travelled by people
who cycle is limited but has grown over the past decade owing in part to
the availability of new data technologies. Researchers have used a
variety of methods to reveal the preferences of cyclists including data
obtained from global positioning system (GPS) devices or smartphone
applications (Pritchard 2018). In general, studies using such data
confirm that people who cycle prefer routes with separated facilities
over mixing with traffic (Chen, Shen, and Childress 2018; Dill 2009;
El-Assi, Mahmoud, and Habib 2017; Lu, Scott, and Dalumpines 2018;
Skov-Petersen et al. 2018) and incorporate infrastructure as part of
their routes (Dill 2009; Lu, Scott, and Dalumpines 2018; Pritchard,
Bucher, and Frøyen 2019). One study conducted in Portland, Oregon using
GPS data found that streets with bike lanes were comparable in
attractiveness to streets with low traffic volume (Broach, Dill, and
Gliebe 2012). By examining GPS data from Hamilton's bicycle-sharing
program, Lu et al.~(2018) found that bike share users travel routes that
are significantly longer than the shortest path distance and are more
likely to use local streets with low traffic and bicycle facilities.
Similarly, Chen et al.~(2018) also reported that people who travel by
bicycle in Seattle, Washington prefer short and flat routes with
connected facilities on roads that have low traffic speeds. Their study
found more variability with respect to preference for views along routes
with features like mixed land use, street trees, lighting, and city
features.

Cycling facilities and street connectivity have most consistently been
found to be an important attribute of the built environment for
promoting cycling (Yang et al. 2019). However, few studies incorporate
variables at two or more spatial scales, as outlined in Moudon and Lee's
(2003) framework to capture a comprehensive view of the variability in
the built environment that a cyclist might encounter. Winters et al.'s
(2010) study in Vancouver, Canada is an exception, as is a recent study
conducted by Cole-Hunter et al.~(2015) that took some factors at the
route level into account in their analysis of cycling propensity in
Barcelona, Spain. Nielsen and Skov-Petersen (2018) recently analyzed the
influence of built environment attributes at three different scales on
the probability of cycling in Copenhagen, Denmark which captured some of
the spatial differentiation at which variables are important, however
they did not include any route analysis. There is a need for more
research to measure and understand the built environment attributes that
affect cycling along different parts of the trip and at different
spatial zones.

\hypertarget{sec:method}{%
\section{Method}\label{sec:method}}

\hypertarget{model}{%
\subsection{Spatial Interaction Models}\label{model}}

We use spatial interaction methods to analyze bicycle trip flows in
Hamilton, Ontario. In the form of a gravity model, this modelling
approach can account for multiple spatial zones along a cycling trip
(see Winters et al. 2010), and is therefore a more holistic approach
than trip generation analysis (e.g.~Dios Ortúzar and Willumsen 2011,
chap. 5). The \emph{Transportation Tomorrow Survey} provides sufficient
information to infer the zone of origin and destination of all bicycle
trips in Hamilton using centroids of the traffic zones. Built
environment attributes at the zone of origin and zone of destination of
cycling trips can be accessed through publicly available data. Finally,
new algorithms for cycle routing like \emph{CycleStreets} now make it
possible to infer route characteristics between origins and
destinations, which can be considered when calculating the trip
distances.

Spatial interaction models operate on principles of propulsion,
attraction, and the friction of space. In other words, we can assume
that there are factors within a particular geographic area that
contribute to producing bicycle trips, such as residential or population
density, and there are factors in other geographic areas that attract
them like jobs or amenities. Finally, there is the friction of space, in
other words, the cost incurred in reaching a destination from an origin.
Spatial interaction models are useful for estimating or explaining
spatial flows in a particular system or to predict them in different
scenarios.

The equation of the spatial interaction model: \[
U_{ij} = f(V_i, Wj, d_{ij})
\]

\noindent where \(i\) represents the origin, \(j\) represents the
destination, \(U_{ij}\) is the total interaction between origin and
destination (i.e., for this analysis it is the number of bicycle trips
recorded in the \emph{TTS}), \(V_i\) is a vector of attributes at the
zone of origin (i.e., the push factors), \(W_j\) is a vector of
attributes at the zone of destination (i.e., the pull factors), and
\(d_{ij}\) represents the cost of making the trip (i.e., often the
distance or time as a measurement of spatial separation).

Poisson regression is commonly used in the estimation of a spatial
interaction model when the dependent variable is available as a count
(Chun 2008; Griffith 2011; Metulini, Patuelli, and Griffith 2018). This
regression model is also suitable for datasets that contain many zero
flows (Griffith 2011) as is the case where many zones of origin and
destination did not generate trips. For our analysis, bicycle trip
counts serve as the dependent variable and built environment or
demographic attributes known to influence cycling serve as independent
variables.

The Poisson regression model can be written in linear form as: \[
ln(\mu_{ij}) = \lambda + \lambda_O ln(V_i) + \lambda_Dln(W_j) + \beta ln(d_{ij})
\]

\noindent where \(\mu_{ij}\) is the number of bicycle trips between zone
of origin \(i\) and zone of destination \(j\), \(V_i\) and \(W_j\)
represent the push and pull factors at \(i\) and \(j\) respectively,
\(d_{ij}\) is the cost or separation between the zone of origin and zone
of destination, and \(\lambda\) are estimable parameters.

As highlighted by numerous studies (e.g., Chun 2008; Metulini, Patuelli,
and Griffith 2018), spatial or network autocorrelation can occur in
spatial interaction models, among other things, because of unobservable
factors at the zone of origin or destination that are not included in
the model or a misspecified cost function. Failing to account for
network autocorrelation can lead to unreliable findings or misleading
interpretations of the behaviour modelled (Chun 2008). Recent papers on
modelling spatial interaction have proposed the use of eigenvector
spatial filtering as a way of accounting for network autocorrelation
(Chun 2008; Griffith 2011; Metulini, Patuelli, and Griffith 2018). In
this respect, use of Moran's \(I\) has been criticized for the case of
residuals of a Poisson regression model because it is based on a
normality assumption and Poisson has distributional properties that are
not well known (Chun 2008). Instead the \(T\) statistic (Jacqmin-Gadda
et al. 1997), is recommended for applications in spatial interaction
modelling (Chun 2008; Metulini, Patuelli, and Griffith 2018). An
important assumption to note is that the residuals of the model are
random and uncorrelated.

\hypertarget{study-area}{%
\subsection{Study Area}\label{study-area}}

Hamilton is a growing mid-sized city located in the Greater Toronto and
Hamilton Area, in Canada. The city is divided by the Niagara Escarpment,
which separates the lower city and downtown core in Dundas Valley from
the suburban/rural parts of the city on top of the escarpment and is
approximately 100m tall in many places. The population was approximately
540,000 in 2016 at the time that the \emph{Transportation Tomorrow
Survey} was conducted, but is expected to increase by 22.9\% over the
coming 15 years (City of Hamilton 2018a), indicating that transportation
demand will likely also grow. The city's current Cycling Master Plan was
released in 2009 to ``guide the development and operation of {[}the
city's{]} cycling infrastructure for the next twenty years'' (City of
Hamilton 2018c, i) and was most recently updated in 2018 (City of
Hamilton 2018b). According to the City of Hamilton, approximately 46\%
of the planned city-wide cycling infrastructure, which includes
on-street and off-street facilities, has been built as of 2019. Around
15 to 20 km of new cycling facilities are built each year, amounting to
an annual increase of 1-2\% for the entire network.

\begin{figure}
\centering
\includegraphics{BE-correlates-cycling-flows_files/figure-latex/context-plot-1.pdf}
\caption{\label{fig:context-plot} Wards in the city of Hamilton (Note:
The smaller zones are traffic zones that generated at least one bicycle
trip and are used in the analysis; black lines are provincial highways
and green line is the Niagara Escarpment)}
\end{figure}

\begin{figure}
\centering
\includegraphics{BE-correlates-cycling-flows_files/figure-latex/trips-by-origin-1.pdf}
\caption{\label{fig:trips-by-origin} Number of Trips Produced by Each
Traffic Zone (Black lines are provincial highways and green line is the
Niagara Escarpment)}
\end{figure}

\hypertarget{data-sources}{%
\subsection{Data Sources}\label{data-sources}}

The \emph{Transportation Tomorrow Survey} (\emph{TTS}) is a voluntary
travel survey conducted every 5 years since 1986 to collect information
about urban travel for commuting purposes in Southern Ontario (Data
Management Group 2018). The final dataset for the 2016 survey includes
6,424 completed surveys in Hamilton out of a total of 162,708 from the
entire GTHA. The results from respondents in Hamilton serve as the
primary dataset used in this analysis and were made available in Spring
2018. The \emph{TTS} study employed a mixed sampling approach that was
primarily address-based in response to changes in landline ownership and
increasing households that only have a cell phone and no landline (Data
Management Group 2018). The survey includes sampling weights to obtain
population-level values of the variables (Data Management Group 2018).
The survey was conducted between September and December 2016 online
(64\% of surveys completed) and by telephone (36\% of surveys
completed). Each participant was asked to provide household and
demographic data (e.g., household size, number of vehicles, gender,
etc.) and to describe all trips (e.g., origin, destination, transport
mode, etc.) made the previous day by each member of the household aged
11 years or older. Trip data are aggregated for public use and the
\emph{traffic zone} is the finest level of spatial disaggregation.
Hamilton has 234 traffic zones. Each traffic zone typically falls along
the centre line of roads or the natural geographic boundaries, but may
or may not align with municipal ward boundaries.

In total, there are 13,635 bicycle trips in the 113 traffic zones within
Hamilton, after the expansion factor to make it a representative sample.
The trips occurred between a total of 294 origin-destination pairs. The
true origins and destinations of trips are not included in the dataset,
only the number of trips producted or attracted to each \emph{traffic
zone}. Although cycling increased overall in the city in recent years,
levels vary across different parts of the city. The geographical context
for the analysis can be seen in Figure \ref{fig:context-plot}, which
shows the Wards in the city (each Ward has an elected representative in
the City Council). Wards 1, 2, and 3, which include the local university
and downtown core, produce the largest numbers of trips by bicycle in
Hamilton. The maximum number of bicycle trips recorded to or from a
traffic zone was 365 trips. This traffic zone is located in Ward 1,
which has the highest cycling mode share in the city, and predominantly
features the local university. This aligns with Scott and Ciuro's (2019)
findings that the university is a major generator and attractor of bike
share trips. Due to low density and few destinations within a bikeable
distance, the majority of traffic zones located in the rural areas of
the city generated 0 bicycle trips. An average of 46 bicycle trips
occurred per traffic zone that produced bicycle trips. The minimum
number of bicycle trips recorded in a traffic zone that produced any
trips at all was 6. Of the 113 traffic zones that produced bicycle
trips, about 25\% produced more than 55 trips, likely zones that feature
attributes that are conducive to greater cycling levels, such
infrastructure or mixed land uses.

Objectively measured demographic and environmental attributes at the
zones of origin and destination that might explain the production or
attraction of bicycle trips were included in the model. These
explanatory variables were selected based on their known or potential
influence on cycling behaviour, as identified in the literature above,
but also on our ability to access such data. For instance, residential
density at the zone of origin might explain why trips begin there, and
the number of jobs or services at the zone of destination could explain
why trips end there. When possible, the datasets used for this analysis
come from 2016 to match the year of the \emph{Transportation Tomorrow
Survey} results. The \emph{2016 Canadian Census}, which is publicly
available information, provided population estimates at the census tract
level. Land use data was accessed from \emph{Teranet Inc.} and The City
of Hamilton's Department of Planning and Economic Development. The
latter dataset defines all land parcels in the city as well as the type
of land use for each parcel. The 2016 \emph{Enhanced Points of Interest}
(EPOI), produced by DMTI Spatial Inc, is a national database of over 1
million business and recreational points of interest in Canada that
featured over 32,000 points of interest located in Hamilton. Finally,
The City of Hamilton's \emph{Open Data Program} offered a dataset
containing the number of transit stops and the number of existing and
proposed cycling infrastructure segments.

\hypertarget{data-preparation}{%
\subsection{Data Preparation}\label{data-preparation}}

Hamilton's bicycle trip records were accessed in July 2019 and exported
as a contingency table with the traffic zones of origin and destination
of all cycling trips. The original table containing only trip
information featured 294 origin-destination (O-D) pairs of traffic
zones. This table was cleaned to remove 13 isolated zones, which
produced trips only to neighbouring zones and not elsewhere in the city.
This reduced the number of O-D pairs in our analysis from 294 to 262.
Objective demographic and environmental variables were geographically
organized in two different zoning systems, and areal interpolation was
performed to convert census data from the tract level to traffic zones.
Similarly, spatial subsetting was performed to select and organize
environmental attributes based on their known coordinates and whether or
not they intersected with a traffic zone. Zonal demographic and
environmental variables were then joined to the origin-destination
table. Table \ref{tab:variables} shows the variables that were tested in
the model to measure their relative and collective influence on cycling
trip flows.

\begin{table}

\caption{\label{tab:data-sources}\label{tab:variables}Demographic and Built Environment Variables Used in the Analysis}
\centering
\resizebox{\linewidth}{!}{
\fontsize{9}{11}\selectfont
\begin{tabular}[t]{>{}l|>{\raggedright\arraybackslash}p{30em}|>{}l}
\toprule
Variable & Description & Source\\
\midrule
\cellcolor{gray!6}{Population} & \cellcolor{gray!6}{Persons residing in each traffic zone (1,000s)} & \cellcolor{gray!6}{2016 Canadian Census}\\
\em{Points of Interest} & Points of interest (e.g., health care and educational facilities, restaurants, etc.) per traffic zone  (1,000s) & DMTI Spatial Inc.\\
\cellcolor{gray!6}{Bus Stops} & \cellcolor{gray!6}{Municipal bus stops per traffic zone  (100s)} & \cellcolor{gray!6}{City of Hamilton Open Data}\\
\em{Infrastructure Segments} & Existing and proposed cycling infrastructure segments (100s) & City of Hamilton Open Data\\
\cellcolor{gray!6}{Institutions} & \cellcolor{gray!6}{Institutions (e.g., schools, places of worship, government, etc.) per traffic zone  (1,000s)} & \cellcolor{gray!6}{Teranet Inc., Hamilton Parcel/Land Use Data}\\
\addlinespace
\em{Commercial} & Commercial locations (e.g., general retail, recreation, and sports clubs, etc.) per traffic zone (1,000s) & Teranet Inc., Hamilton Parcel/Land Use Data\\
\cellcolor{gray!6}{Residential} & \cellcolor{gray!6}{Residences (e.g., detached house, semi-detached house, apartment, etc.) per traffic zone (1,000s)} & \cellcolor{gray!6}{Teranet Inc., Hamilton Parcel/Land Use Data}\\
\em{Full-Time Jobs} & Persons employed full-time, outside of the home, by zone of employment (1,000s) & Transportation Tomorrow Survey\\
\cellcolor{gray!6}{Part-Time Jobs} & \cellcolor{gray!6}{Persons employed part-time, outside of the home, by zone of employment (1,000s)} & \cellcolor{gray!6}{Transportation Tomorrow Survey}\\
\bottomrule
\end{tabular}}
\end{table}

In addition to the variables in Table \ref{tab:variables}, dummy
variables were created to account for Hamilton's topography. Traffic
zones were classified by geographic area, namely zones in the lower city
and zones in the Niagara Escarpment or the suburban/rural parts of the
city. This classification was used to code O-D pairs that were in the
same different geographical classes, to capture that a cyclist would
need to navigate changes in elevation and natural features when
travelling across different topographies in Hamilton. If both the zone
of origin and zone of destination were in the lower city, this pair was
labelled with 0. If the origin and destination were in different regions
(i.e., lower city and escarpment/rural), the pair was labelled with 1.
If both zones were in the escarpment or rural areas, the pair was
labelled with 2.

\hypertarget{inferring-cycle-routes}{%
\subsection{Inferring Cycle Routes}\label{inferring-cycle-routes}}

The \emph{Transportation Tomorrow Survey} does not ask respondents to
state the routes that they travel, so this information is unknown. For
this reason, we have to infer routes using the centroids of each traffic
zone polygon as a start or end point, which are then included as a cost
function in the model. We use a novel routing service,
\emph{CycleStreets}\footnote{https://www.cyclestreets.net/}, to this
end. The algorithm relies on data that is publicly available through
\emph{OpenStreetMap}, so there are additional objectively measured
environmental variables captured in the cost function. This can
potentially provide more information about trip distribution than using
only the shortest-path distance. The algorithm infers three different
types of routes: \emph{fastest}, \emph{quietest}, and \emph{balanced}.
The \texttt{R}
package\footnote{https://cran.r-project.org/web/packages/cyclestreets/cyclestreets.pdf}
used in this analysis states: ``These represent routes taken to minimize
time, avoid traffic, and compromise between the two, respectively''
(Lovelace and Lucas-Smith 2018, 1). The \emph{CycleStreets} algorithm
rates the \emph{quietness} as a score, with routes featuring cycle
tracks and park paths rated as the quietest, and then decreasing to
varying degrees of quietness depending on the extent that cyclists would
have to interact with other users of the road
on\footnote{https://www.cyclestreets.net/help/journey/howitworks/\#quietness}.
With respect to the algorithm, the documentation explains that routes
with shared facilities rate relatively high and busy roads have the
lowest score. Overall, the algorithm tries to minimize the
\emph{busyness} of a route but there is a lack of transparency with
respect to the rate of speed used for calculating the time of each route
and which specific attributes are considered by the algorithm when
minimizing \emph{busyness}. Quietness scores are adjusted based on
feedback from
users\footnote{\url{https://www.cyclestreets.net/help/journey/howitworks/}},
and therefore include what might be considered expert opinion. The
scores are used to determine whether a route is \emph{fastest},
\emph{balanced}, or \emph{quietest}. The distance and time on each leg
of a route can be obtained from the algorithm, and from these, the total
travel distance and time for each type of route between origin and
destination can be calculated.

Testing each type of route as an impedance factor in the model yields
six different cost variables for each origin-destination pair (i.e.,
\(fastest-distance\), \(fastest-time\), \(quietest-distance\),
\(quietest-time\), \(balanced-distance\), and \(balanced-time\)). For
the sake of comparison, we also include the simplest measure of cost,
which is the Euclidean distance between origin-destination centroids (or
Euclidean time, the time that it would take to travel that distance on a
straight line, assuming a speed of 22.5 km/h). Each of these variables
were incorporated into the spatial interaction model to test which cost
variable best explains cycling flows in Hamilton. Table \ref{tab:routes}
offers descriptive statistics of the different types of routes, after
removing intrazonal trips. Table \ref{tab:detours} includes the average
detour of the quietest and balanced routes compared to the Euclidean
distance. The detour is defined as the ratio of the distance (or time)
on the route to the Euclidean distance (or time) for the same
origin-destination pair. For example, a detour of 1.5 means that the
route is 50\% longer than the corresponding Euclidean metric.

\begin{table}

\caption{\label{tab:route-sources}\label{tab:routes}Descriptive Statistics of Inferred Routes by CycleStreets}
\centering
\resizebox{\linewidth}{!}{
\fontsize{9}{11}\selectfont
\begin{tabular}[t]{>{}l|>{\raggedright\arraybackslash}p{5em}|>{}llllll}
\toprule
Route & Minimum & Quartile.1 & Median & Mean & Quartile.3 & Max & SD\\
\midrule
\cellcolor{gray!6}{Euclidean Distance (km)} & \cellcolor{gray!6}{0.318} & \cellcolor{gray!6}{3.387} & \cellcolor{gray!6}{5.508} & \cellcolor{gray!6}{5.924} & \cellcolor{gray!6}{7.950} & \cellcolor{gray!6}{19.631} & \cellcolor{gray!6}{3.367}\\
\em{Euclidean Time (min)} & 0.8461 & 9.0188 & 14.668 & 15.775 & 21.172 & 52.279 & 8.968\\
\cellcolor{gray!6}{Quietest Distance (km)} & \cellcolor{gray!6}{0.412} & \cellcolor{gray!6}{4.950} & \cellcolor{gray!6}{7.944} & \cellcolor{gray!6}{8.293} & \cellcolor{gray!6}{11.085} & \cellcolor{gray!6}{25.523} & \cellcolor{gray!6}{4.419}\\
\em{Quietest Time (mins)} & 1.617 & 22.725 & 37.817 & 40.572 & 54.683 & 133.117 & 4.373\\
\cellcolor{gray!6}{Balanced Distance (km)} & \cellcolor{gray!6}{0.412} & \cellcolor{gray!6}{4.829} & \cellcolor{gray!6}{7.688} & \cellcolor{gray!6}{8.127} & \cellcolor{gray!6}{10.784} & \cellcolor{gray!6}{24.908} & \cellcolor{gray!6}{4.424}\\
\addlinespace
\em{Balanced Time (mins)} & 1.617 & 20.837 & 34.825 & 36.752 & 49.462 & 124.567 & 23.091\\
\cellcolor{gray!6}{Fastest Distance (km)} & \cellcolor{gray!6}{0.412} & \cellcolor{gray!6}{4.851} & \cellcolor{gray!6}{7.715} & \cellcolor{gray!6}{8.179} & \cellcolor{gray!6}{10.834} & \cellcolor{gray!6}{24.865} & \cellcolor{gray!6}{20.376}\\
\em{Fastest Time (mins)} & 1.617 & 20.038 & 32.975 & 34.612 & 46.783 & 110.300 & 18.788\\
\bottomrule
\end{tabular}}
\end{table}

\begin{table}

\caption{\label{tab:unnamed-chunk-1}\label{tab:detours}Descriptive Statistics of Average Detour of Inferred Routes by CycleStreeets Compared to Euclidean Distance}
\centering
\resizebox{\linewidth}{!}{
\fontsize{9}{11}\selectfont
\begin{tabular}[t]{>{}l|>{\raggedright\arraybackslash}p{5em}|>{}lllll}
\toprule
Route & Min & Quartile.1 & Median & Mean & Quartile.3 & Max\\
\midrule
\cellcolor{gray!6}{Quietest Distance (km)} & \cellcolor{gray!6}{0.9861} & \cellcolor{gray!6}{1.2782} & \cellcolor{gray!6}{1.3971} & \cellcolor{gray!6}{1.4388} & \cellcolor{gray!6}{1.5187} & \cellcolor{gray!6}{5.4321}\\
\em{Quietest Time (mins)} & 1.058 & 2.108 & 2.403 & 2.661 & 2.982 & 15.943\\
\cellcolor{gray!6}{Balanced Distance (km)} & \cellcolor{gray!6}{0.9861} & \cellcolor{gray!6}{1.2574} & \cellcolor{gray!6}{1.3794} & \cellcolor{gray!6}{1.4064} & \cellcolor{gray!6}{1.4893} & \cellcolor{gray!6}{5.4149}\\
\em{Balanced Time (mins)} & 1.058 & 1.960 & 2.182 & 2.421 & 2.689 & 15.791\\
\bottomrule
\end{tabular}}
\end{table}

As seen in Tables \ref{tab:routes} and \ref{tab:detours} \emph{quietest}
distance routes and \emph{quietest} time routes are longer than the
\emph{balanced} and \emph{fastest} route counterparts, but not by much.
Most of the \emph{quietest} distance routes are also 50\% longer than
the Euclidean distance.

\hypertarget{sec:results}{%
\section{Results}\label{sec:results}}

\hypertarget{spatial-interaction-models-considered}{%
\subsection{Spatial Interaction Models
Considered}\label{spatial-interaction-models-considered}}

Four spatial interaction models were estimated with bicycle flows
between zones of origin and destination as the dependent variable.
Various combinations of zonal attributes and the distance or time of
inferred cycle routes between origins and destinations were experimented
with. Each of these models went through a general-to-particular variable
selection process. Starting with models that included all zonal
attributes in Table \ref{tab:variables}, variables that did not meet a
significance criterion of \(p \le 5\)\% were removed to obtain a more
parsimonious model. For comparison purposes, a base model with a
constant only was estimated to serve as a benchmark. This was followed
by a model with only zonal attributes (i.e., push-pull factors), then a
model only with cost variables (time or distance of different inferred
routes), and then finally a full model with zonal and cost variables.
The selection of initial variables for each model was deliberate and
meant to investigate the performance of models that considered only
certain aspects of the spatial interaction process. The models are
described next and the results are presented in Table
\ref{tab:all-models}.

\hypertarget{model-1-zonal-attributes-only}{%
\subsubsection{Model 1: Zonal Attributes
Only}\label{model-1-zonal-attributes-only}}

After the benchmark non-informative model, the first estimated spatial
interaction model included zonal attributes as explanatory variables
that might explain the production or attraction of bicycle trips but did
not include a cost variable. These were the variables that met the
significance criterion of \(p \le 5\)\% in the general-to-particular
variable selection process. Therefore, this model did not include the
second spatial zone of the behavioral model of the environment.

\hypertarget{model-2-cost-variables-only}{%
\subsubsection{Model 2: Cost Variables
Only}\label{model-2-cost-variables-only}}

This model used only cost variables, which included our geographical
classes for the zones. In other words, this model includes only
attributes of the second spatial zone of the \emph{behavioral model of
the environment}, namely the different inferred routes. We estimated
this model with one cost variable at a time (e.g., topography and
\emph{fastest} distance, topography and \emph{quietest} time, and so
forth) which allowed for the comparison of how distance or time along
specific inferred routes performed in each model.

\hypertarget{model-3-full-model}{%
\subsubsection{Model 3: Full Model}\label{model-3-full-model}}

In the final model, we combined the variables used in Models 1 and 2, to
include both zonal attributes and the cost function. This model includes
zonal attributes that might explain the production or attraction of
bicycle trips, topography classification, and measures of cost from
inferred routes. Just like Model 2, we estimated this model using each
cost variable at the time (i.e., all zonal attributes with
\emph{fastest} distance as cost, and so forth).

\hypertarget{model-results}{%
\subsection{Model Results}\label{model-results}}

Akaike's information criterion (\(AIC\)) is used to compare the various
models. The model with the lowest \(AIC\) is selected as the model that
minimizes information loss, while considering parsimony of the
specification. In addition, \(AIC\) is used in the calculation of the
relative likelihood, which is defined as: \[
e^{\frac{AIC_{min} - AIC_i}{2}}
\]

In the above, \(AIC_{min}\) is the \(AIC\) of the model that minimizes
this criterion, and \(AIC_i\) is the \(AIC\) of a competing model. This
is measure of of goodness of fit is interpreted as the probability that
the competing model minimizes information loss to the same extent as the
best model. It is important to note that although comparison of \(AIC\)
from a set of models indicates the model with the best fit, it does not
reveal any information about the quality of each model, which is why
analysis of the residuals is important as well.

Table \ref{tab:model-comparison} presents a summary of the goodness of
fit of the models. For reference, the \(AIC\) of the base model is
95,808 and the \(AIC\) of Model 1 is 83,995. Model 1 is a significantly
better fit than the base model, which indicates the explanatory power of
zonal attributes as independent variables. Interestingly, as seen in the
table, the use of cost as in Model 2, provides much higher explanatory
power than zonal attributes. Of the different cost variables, distance
along \emph{quietest} routes is the cost variable that leads to the best
fit, a result that is replicated in Model 3. An obvious limitation of
Model 2 is that it lacks variables that might ultimately explain what is
producing or attracting bicycle trips from each traffic zone. Our full
model, Model 3, includes variables that might explain trips and cost
variables, which ultimately provides the best fit of all models
considered. As seen in Table \ref{tab:model-comparison}, Model 3 with
distance along inferred \emph{quietest} routes provides a significantly
better fit than any of the competing models, and the relative likelihood
(calculated with respect to this model) indicates that the probability
that any of the alternative models minimizes the information loss to the
same extent is practically zero.

\begin{table}

\caption{\label{tab:model-comparison}\label{tab:model-comparison} Model Comparison: AIC and Relative Likelihood}
\centering
\resizebox{\linewidth}{!}{
\begin{tabular}[t]{lcccc}
\toprule
\multicolumn{1}{c}{} & \multicolumn{2}{c}{Model 2} & \multicolumn{2}{c}{Model 3} \\
\cmidrule(l{3pt}r{3pt}){2-3} \cmidrule(l{3pt}r{3pt}){4-5}
Cost Variable & AIC & Relative Likelihood & AIC & Relative Likelihood\\
\midrule
\cellcolor{gray!6}{Euclidean Distance} & \cellcolor{gray!6}{71514} & \cellcolor{gray!6}{<0.0001} & \cellcolor{gray!6}{63127} & \cellcolor{gray!6}{<0.0001}\\
Fastest Distance & 71839 & <0.0001 & 63365 & <0.0001\\
\cellcolor{gray!6}{Fastest Time} & \cellcolor{gray!6}{71969} & \cellcolor{gray!6}{<0.0001} & \cellcolor{gray!6}{63521} & \cellcolor{gray!6}{<0.0001}\\
Quietest Distance & 71307 & <0.0001 & 62973 & 1\\
\cellcolor{gray!6}{Quietest Time} & \cellcolor{gray!6}{71589} & \cellcolor{gray!6}{<0.0001} & \cellcolor{gray!6}{63132} & \cellcolor{gray!6}{<0.0001}\\
\addlinespace
Balanced Distance & 71647 & <0.0001 & 63357 & <0.0001\\
\cellcolor{gray!6}{Balanced Time} & \cellcolor{gray!6}{71979} & \cellcolor{gray!6}{<0.0001} & \cellcolor{gray!6}{63541} & \cellcolor{gray!6}{<0.0001}\\
\bottomrule
\multicolumn{5}{l}{\rule{0pt}{1em}\textit{Note: }}\\
\multicolumn{5}{l}{\rule{0pt}{1em}Relative likelihood is calculated with respect to Model 3:Quietest Distance}\\
\end{tabular}}
\end{table}

The results of the models are presented in Table \ref{tab:all-models}.
In addition to their goodness of fit, each of the models was tested for
network autocorrelation, using Jacqmin-Gadda's \(T\) statistic
(Jacqmin-Gadda et al. 1997; Chun 2008; Metulini, Patuelli, and Griffith
2018). Network autocorrelation is, in addition to a violation of the
independence assumption, an indication of a model that is misspecified
(either the functional form in incorrect or there are relevant variables
that were omitted).

It is worth noting that the only model without residual network
autocorrelation is Model 3, which signifies that this model not only
provides the best fit but it is also the only one that is free from
network autocorrelation. As described above, testing for network
autocorrelation in a spatial interaction model is a diagnostic tool.
When no network autocorrelation is detected in the residuals of the
model, this is a sign that all systematic variation has been accounted
for with the variables included in the model. The model can be
considered a \emph{sufficient} explanation of the pattern observed. We
discuss the results of the analysis next.

\begin{landscape}\begin{table}

\caption{\label{tab:all-models}\label{tab:all-models} Results of the Models}
\centering
\resizebox{\linewidth}{!}{
\begin{tabular}[t]{lcccccccc}
\toprule
\multicolumn{1}{c}{} & \multicolumn{2}{c}{Base Model} & \multicolumn{2}{c}{Model 1} & \multicolumn{2}{c}{Model 2} & \multicolumn{2}{c}{Model 3} \\
\cmidrule(l{3pt}r{3pt}){2-3} \cmidrule(l{3pt}r{3pt}){4-5} \cmidrule(l{3pt}r{3pt}){6-7} \cmidrule(l{3pt}r{3pt}){8-9}
Variable & Estimate & p-value & Estimate & p-value & Estimate & p-value & Estimate & p-value\\
\midrule
\cellcolor{gray!6}{(Intercept)} & \cellcolor{gray!6}{0.2099} & \cellcolor{gray!6}{< 0.0001} & \cellcolor{gray!6}{0.0396} & \cellcolor{gray!6}{0.3735} & \cellcolor{gray!6}{2.4061} & \cellcolor{gray!6}{< 0.0001} & \cellcolor{gray!6}{1.9826} & \cellcolor{gray!6}{< 0.0001}\\
Population.o &  &  & -93.5856 & < 0.0001 &  &  & -27.3106 & 0.0097\\
\cellcolor{gray!6}{Points\_of\_Interest.o} & \cellcolor{gray!6}{} & \cellcolor{gray!6}{} & \cellcolor{gray!6}{-0.6827} & \cellcolor{gray!6}{< 0.0001} & \cellcolor{gray!6}{} & \cellcolor{gray!6}{} & \cellcolor{gray!6}{-0.7509} & \cellcolor{gray!6}{< 0.0001}\\
Institutions.o &  &  & 30.4358 & < 0.0001 &  &  & 14.6119 & < 0.0001\\
\cellcolor{gray!6}{Commercial.o} & \cellcolor{gray!6}{} & \cellcolor{gray!6}{} & \cellcolor{gray!6}{5.5309} & \cellcolor{gray!6}{< 0.0001} & \cellcolor{gray!6}{} & \cellcolor{gray!6}{} & \cellcolor{gray!6}{-1.6004} & \cellcolor{gray!6}{0.0011}\\
Industry.o &  &  & -5.8369 & < 0.0001 &  &  & 1.6853 & 0.0125\\
\cellcolor{gray!6}{Office.o} & \cellcolor{gray!6}{} & \cellcolor{gray!6}{} & \cellcolor{gray!6}{-11.2755} & \cellcolor{gray!6}{< 0.0001} & \cellcolor{gray!6}{} & \cellcolor{gray!6}{} & \cellcolor{gray!6}{-10.2228} & \cellcolor{gray!6}{< 0.0001}\\
Residential.o &  &  & -0.7102 & < 0.0001 &  &  & -0.1558 & < 0.0001\\
\cellcolor{gray!6}{BusStops.o} & \cellcolor{gray!6}{} & \cellcolor{gray!6}{} & \cellcolor{gray!6}{1.6269} & \cellcolor{gray!6}{< 0.0001} & \cellcolor{gray!6}{} & \cellcolor{gray!6}{} & \cellcolor{gray!6}{1.7688} & \cellcolor{gray!6}{< 0.0001}\\
BikeInfra.o &  &  & 0.0107 & < 0.0001 &  &  & 0.0019 & 0.0926\\
\cellcolor{gray!6}{Population.d} & \cellcolor{gray!6}{} & \cellcolor{gray!6}{} & \cellcolor{gray!6}{-166.7009} & \cellcolor{gray!6}{< 0.0001} & \cellcolor{gray!6}{} & \cellcolor{gray!6}{} & \cellcolor{gray!6}{-97.1404} & \cellcolor{gray!6}{< 0.0001}\\
Institutions.d &  &  & 36.4256 & < 0.0001 &  &  & 24.1953 & < 0.0001\\
\cellcolor{gray!6}{Commercial.d} & \cellcolor{gray!6}{} & \cellcolor{gray!6}{} & \cellcolor{gray!6}{1.5804} & \cellcolor{gray!6}{7e-04} & \cellcolor{gray!6}{} & \cellcolor{gray!6}{} & \cellcolor{gray!6}{-8.0098} & \cellcolor{gray!6}{< 0.0001}\\
Industry.d &  &  & -6.0468 & < 0.0001 &  &  & 5.3459 & < 0.0001\\
\cellcolor{gray!6}{Office.d} & \cellcolor{gray!6}{} & \cellcolor{gray!6}{} & \cellcolor{gray!6}{6.8574} & \cellcolor{gray!6}{< 0.0001} & \cellcolor{gray!6}{} & \cellcolor{gray!6}{} & \cellcolor{gray!6}{15.0947} & \cellcolor{gray!6}{< 0.0001}\\
Residential.d &  &  & 0.1398 & < 0.0001 &  &  & 0.7039 & < 0.0001\\
\cellcolor{gray!6}{BusStops.d} & \cellcolor{gray!6}{} & \cellcolor{gray!6}{} & \cellcolor{gray!6}{-1.503} & \cellcolor{gray!6}{< 0.0001} & \cellcolor{gray!6}{} & \cellcolor{gray!6}{} & \cellcolor{gray!6}{-1.2592} & \cellcolor{gray!6}{< 0.0001}\\
BikeInfra.d &  &  & 0.0024 & 0.0442 &  &  & -0.0083 & < 0.0001\\
\cellcolor{gray!6}{Full\_time\_jobs.d} & \cellcolor{gray!6}{} & \cellcolor{gray!6}{} & \cellcolor{gray!6}{0.1907} & \cellcolor{gray!6}{< 0.0001} & \cellcolor{gray!6}{} & \cellcolor{gray!6}{} & \cellcolor{gray!6}{0.0796} & \cellcolor{gray!6}{< 0.0001}\\
Part\_time\_jobs.d &  &  & 0.2368 & < 0.0001 &  &  & 0.5414 & < 0.0001\\
\cellcolor{gray!6}{Topographylower city - rural} & \cellcolor{gray!6}{} & \cellcolor{gray!6}{} & \cellcolor{gray!6}{} & \cellcolor{gray!6}{} & \cellcolor{gray!6}{-2.53} & \cellcolor{gray!6}{< 0.0001} & \cellcolor{gray!6}{-2.4021} & \cellcolor{gray!6}{< 0.0001}\\
Topographyrural &  &  &  &  & -0.6518 & < 0.0001 & -0.545 & < 0.0001\\
\cellcolor{gray!6}{quietest\_distance} & \cellcolor{gray!6}{} & \cellcolor{gray!6}{} & \cellcolor{gray!6}{} & \cellcolor{gray!6}{} & \cellcolor{gray!6}{-0.3253} & \cellcolor{gray!6}{< 0.0001} & \cellcolor{gray!6}{-0.345} & \cellcolor{gray!6}{< 0.0001}\\
\addlinespace[2em]
\multicolumn{9}{l}{\textbf{Model diagnostics}}\\
\hspace{1em}Jacqmin-Gadda z(T) & 34.4011 & p < 0.0001 & 26.5433 & p < 0.0001 & 28.5666 & p < 0.0001 & 0.1576 & p =  0.4374\\
\cellcolor{gray!6}{n =} & \cellcolor{gray!6}{} & \cellcolor{gray!6}{9801} & \cellcolor{gray!6}{} & \cellcolor{gray!6}{9801} & \cellcolor{gray!6}{} & \cellcolor{gray!6}{9801} & \cellcolor{gray!6}{} & \cellcolor{gray!6}{9801}\\
\hspace{1em}log-likelihood = &  & -47902.9293 &  & -41976.9929 &  & -35649.4031 &  & -31463.3543\\
\cellcolor{gray!6}{AIC =} & \cellcolor{gray!6}{} & \cellcolor{gray!6}{95807.8587} & \cellcolor{gray!6}{} & \cellcolor{gray!6}{83993.9858} & \cellcolor{gray!6}{} & \cellcolor{gray!6}{71306.8063} & \cellcolor{gray!6}{} & \cellcolor{gray!6}{62972.7087}\\
\hspace{1em}Relative likelihood = &  & < 0.0001 &  & < 0.0001 &  & < 0.0001 &  & 1\\
\bottomrule
\multicolumn{9}{l}{\rule{0pt}{1em}\textit{Note: }}\\
\multicolumn{9}{l}{\rule{0pt}{1em}Jacqmin-Gadda $T$ is converted to a $z$-score}\\
\multicolumn{9}{l}{\rule{0pt}{1em}Relative likelihood is calculated with respect to Model 3}\\
\end{tabular}}
\end{table}
\end{landscape}

\hypertarget{sec:discussion}{%
\section{Discussion}\label{sec:discussion}}

\hypertarget{best-fit-model}{%
\subsection{Best Fit Model}\label{best-fit-model}}

Model 3 reveals that several built environment attributes at the zones
of origin and destination produce or attract bicycle trips in Hamilton,
Ontario. Points of interest and commercial and office locations at the
zone of origin had a negative influence on the number of expected
bicycle trips. This is as expected: more destinations or amenities at
the origin create more intervening opportunities that ultimately reduce
the need to travel to other areas. Although population density has been
found to influence cycling trips in several studies (e.g., Nielsen and
Skov-Petersen 2018; Nordengen et al. 2019; Schneider and Stefanich 2015;
Winters et al. 2010), in our analysis we find a negative effect of
population density in terms of both producing and attracting trips by
bicycle. Scott and Ciuro similarly found that population density around
bike share hubs in Hamilton does not influence ridership (Scott and
Ciuro 2019). It is possible that this is due to the relatively low
population density of Hamilton in general. In contrast, availability of
jobs at the destination was a positive attractor of bicycle trips. The
model also uncovered a positive relationship between number of trips and
different land uses at the destination: institution, industry, office,
residential locations. This reflects an abundance of amenities and
diversity of jobs, as well as the reciprocal trip flow to return to
one's residence. Geographical classification of the zones was found to
have a negative relationship with the number of bicycle trips. This
suggests relatively little interaction between the two broad regions in
the city, namely lower city and escarpment/suburban/rural, and also
lower interaction within the escarpment/suburban/rural compared to the
lower city. The presence of the Niagara Escarpment, in particular,
echoes other studies that have found that elevation at the destination
or changes in slope can deter travel by bicycle (e.g., Broach, Dill, and
Gliebe 2012; Cole-Hunter et al. 2015). In the case of Hamilton, the
Escarpment is a significant change in slope. The physical cost of
travelling up and down an escarpment, in addition to longer trip
distances, can help to explain why there are few trips between the two
distinct areas of the city.

\hypertarget{inferred-routes}{%
\subsection{Inferred Routes}\label{inferred-routes}}

The model reveals that the \emph{quietest} routes that allow cyclists to
minimize distance \emph{and} interactions with other road users best
explain the pattern of travel by bicycle in Hamilton. This suggests that
people who travel by bicycle in Hamilton likely seek out routes that are
less busy with car traffic instead of more direct routes. This finding
is consistent with previous research that used GPS data to reveal the
route preferences of cyclists in Hamilton (Lu, Scott, and Dalumpines
2018). After \emph{quietest} distance, \emph{quietest} time was the
closest competitor. After the identified \emph{quietest} routes, there
was relatively little difference between using \emph{balanced} distance
and \emph{fastest} distance as a measure of spatial separation in the
model. Intuitively, it makes sense that these two measures would have
similar goodness of fit since they both involve greater mixing with
traffic. If traffic interactions cannot be avoided, then taking the
fastest shortest path route to arrive at the destination would likely be
the next best option for many.

Inferring the route travelled between zones of origins and destination
using routing algorithms is an important method to account for route
characteristics in the analysis of trip distribution, and the second
spatial zone of the \emph{behavioral model of the environemnt}. It is
also useful given that such data about routes is not available from
travel surveys, including the \emph{Transportation Tomorrow Survey}.
Other studies have used GIS (Cole-Hunter et al. 2015; Winters et al.
2010), GPS data (Chen, Shen, and Childress 2018; Dill 2009; Lu, Scott,
and Dalumpines 2018; Skov-Petersen et al. 2018), or new methods using
crowd-sourced data (McArthur and Hong 2019; Sarjala 2019) to measure or
approximate the built environment along routes travelled by cyclists.
However, GPS data are typically available only for small samples or
under limited conditions, such as with bike share trips in Hamilton (Lu,
Scott, and Dalumpines 2018), that may not cover the full geographical
extent of travel by bicycle in a region. To the best of our knowledge,
this is the first North American study that uses the \emph{CycleStreets}
algorithm in combination with travel survey data to infer routes in the
analysis of cycling trip flows. As previously noted, the differences
between \emph{quietest} routes and \emph{balanced}/\emph{fastest} routes
are relatively small. The fact that the goodness of fit of the models
using the \emph{quietest} routes is significantly better suggests that
there might be other factors at the micro-level of the routes that may
influence cycling differently between these routes. This leaves open the
question whether routes inferred by \emph{CycleStreets} have attributes
that support cycling, such as infrastructure or enjoyable environments,
in addition to less \emph{busyness}. Despite the lack of transparency
about certain aspects of the algorithm (in particular the expert input
used to modify the \emph{quietness} scores), in the experience of the
authors the algorithm makes overall sensible recommendations for
\emph{quietest} routes. While we cannot know with complete certainty
which routes were actually used by individual cyclists, by exploring
different types of routes in our models we are able to provide
statistical support for \emph{quietest} routes that minimize distance -
a finding in line with bicycle-share results reported by Lu et
al.~(2018).

\begin{table}

\caption{\label{tab:unnamed-chunk-5}\label{tab:flow-descriptives}Bicycle Trip Flow Characteristics According to Quietest Distance Routes}
\centering
\fontsize{9}{11}\selectfont
\begin{tabular}[t]{>{}l|>{\raggedright\arraybackslash}p{5em}|}
\toprule
Characteristics & Percentage\\
\midrule
\cellcolor{gray!6}{\% Trip Flows >= 10 km} & \cellcolor{gray!6}{4.6\%}\\
\em{\% Trip Flows >= 5 km} & 21.4\%\\
\cellcolor{gray!6}{\% Trip Flows <= 5 km} & \cellcolor{gray!6}{78.6\%}\\
\em{\% Trip Flows <= 2.5 km} & 48.5\%\\
\cellcolor{gray!6}{\% Trip Flows Rural/Escarpment to Lower City} & \cellcolor{gray!6}{1.9\%}\\
\addlinespace
\em{\% Trip Flows Lower City to Rural/Escarpment} & 1.9\%\\
\cellcolor{gray!6}{\% Trip Flows Only Rural/Escarpment} & \cellcolor{gray!6}{16.8\%}\\
\em{\% Trip Flows Only Lower City} & 79.4\%\\
\cellcolor{gray!6}{\% Trip Flows <= 2.5 km in Lower City} & \cellcolor{gray!6}{52.4\%}\\
\em{\% Trip Flows <= 5 km in Lower City} & 80.8\%\\
\addlinespace
\cellcolor{gray!6}{\% Trip Flows >= 5 km in Lower City} & \cellcolor{gray!6}{19.2\%}\\
\em{\% Trip Flows <= 2.5 km in Rural/Escarpment Area} & 40.9\%\\
\cellcolor{gray!6}{\% Trip Flows <= 5 km in Rural/Escarpment Area} & \cellcolor{gray!6}{86.4\%}\\
\em{\% Trip Flows >= 5 km in Rural/Escarpment Area} & 13.6\%\\
\bottomrule
\end{tabular}
\end{table}

\hypertarget{analysis-of-residuals}{%
\subsection{Analysis of Residuals}\label{analysis-of-residuals}}

The best model minimized information loss conditional on the independent
variables. Informed by the work of Moniruzzaman and Páez (2012) with
walking trips in Hamilton, we were curious to examine in more detail
over- and under-estimated trip flows. There were a total of 4
over-estimated trip flows and 256 under-estimated trip flows. Since the
model is not Gaussian, there is no assumption that the distribution of
the residuals will be symmetric. We hypothesize that discrepancies
between the number of observed trips and the number of expected trips
are due to the built environment, namely attributes along the
\emph{quietest} distance route that might influence cycling but that we
were not able to capture in the model. With respect to over-estimated
trip flows, there may be barriers along the inferred cycle route between
zone of origin and destination that deter cyclists from travelling. The
opposite may be true for under-estimated trip flows. It is worth noting
first that the majority of trip flows were under-estimated which
indicates, to some extent, that there is more cycling in Hamilton than
predicted by the model. This suggests that route characteristics that
influence cycling may be influential. We provide a qualitative
description of these trip flows next.

By plotting the negative residuals from the best model, after removing
all origin-destination (OD) pairs with zero trips, bicycle trip flows
that were over-estimated were visualized in Figure
\ref{fig:residuals-overestimated}. There were only 4 trip flows, 3 of
which represent travel in a westward direction. The zone of destination
for 3 of the 4 trip flows includes the university which is a major
employment and educational institution, and thus acts as a strong push
and pull factor for trips. This was identified with bike share trips in
Hamilton as well (Scott and Ciuro 2019). Schneider et al.~(2015) also
found that neighbourhoods with high levels of commute trips by bicycle
are located near to a university campus. This suggests that universities
can attract a large number of trips. Upon further investigation, the
\emph{Enhanced Points of Interest} dataset catalogues each different
building and unit within the university, meaning that there are several
hundred destinations within the traffic zone. The count may have skewed
the relative influence of the university by indicating more potential
destinations, instead of one institution, leading to over-estimation.
The zone of destination for the other trip flow was also near the
university, however the over-estimation was almost negligible. When
analyzing the \emph{quietest} distance routes for the O-D pairs that end
at the traffic zone with the university, each route would require a
cyclist to cross a major highway or travel along an arterial road. At
the route level, road networks with fewer highways or arterial roads
have been found to increase the likelihood of making a trip by bicycle
(Winters et al. 2010; Zhao 2014). Although we were able to provide
statistical support for \emph{quietest} routes that minimize distance,
there are still roads and intersections in Hamilton that cannot be
avoided and that still feature along routes that are less busy overall.

\begin{figure}
\centering
\includegraphics{BE-correlates-cycling-flows_files/figure-latex/residuals-over-1.pdf}
\caption{\label{fig:residuals-overestimated} Map of Over-predicted
Bicycle Trip Flows (Black lines are provincial highways and green line
is the Niagara Escarpment)}
\end{figure}

Similarly, by plotting the positive residuals, after removing all
origin-destination pairs with zero trips, bicycle trip flows that were
under-estimated were visualized in Figure
\ref{fig:residuals-underestimated}. Given that the majority of trip
flows were under-estimated, we visualized trip flows in different maps
according to their characteristics. Figure \ref{fig:residuals-over-5km}
shows a map of trip flows over 5 km and Figure
\ref{fig:residuals-under-5km} shows trip flows under 5 km. One fifth of
under-estimated trip flows, approximately 21\%, had a \emph{quietest}
distance route between 5 and 25 kilometres. Trip distance is an
important determinant of cycling for transport (Heinen, Wee, and Maat
2010), which suggests that the distance between origin and destination
could be the reason that these flows were under-estimated. Furthermore,
approximately 17\% of under-estimated trip flows occurred within the
suburban neighbourhoods on the Niagara Escarpment. Fewer cycling trips
were expected in this area of the city because bicycle trips are
typically less likely in low density areas where there are fewer
destinations that can be reached in short distances, as was found to be
the case in the United States (Pucher and Buehler 2006). It is also
worth noting in this case that Hamilton's suburban areas have far less
cycling facilities compared to the lower city, which reinforces the
car-centric design of these neighbourhoods. Finally, there is a
noteworthy cluster of trip flows in the city's downtown core of 5 km or
less. Nielsen and Skov-Petersen (2018) note that built environment
attributes are effective at different spatial scales. They uncovered
positive effects of cycling infrastructure within 1 km of the home on
the probability of cycling, providing evidence that proximity to cycling
facilities can influence transport mode choices (Nielsen and
Skov-Petersen 2018). We hypothesize that this cluster was
under-estimated because cycling infrastructure has been built more
extensively in the downtown core and is likely normalizing travel by
bicycle in this area. The connectivity of such infrastructure between
zone of origin and zone of destination may not have been captured in the
inferred routes using as the cost function, leading to under-estimation.
Likewise, the downtown core features a higher density of destinations
within a 1-5 km distance that Hamiltonians could comfortably travel to
bicycle, compared to single use neighbourhoods.

\begin{figure}
\centering
\includegraphics{BE-correlates-cycling-flows_files/figure-latex/residuals-under-1.pdf}
\caption{\label{fig:residuals-underestimated} Map of Under-predicted
Bicycle Trip Flows (Black lines are provincial highways and green line
is the Niagara Escarpment)}
\end{figure}

\begin{figure}
\centering
\includegraphics{BE-correlates-cycling-flows_files/figure-latex/residuals-over-5km-1.pdf}
\caption{\label{fig:residuals-over-5km} Map of Under-predicted Bicycle
Trip Flows Over 5 km (Black lines are provincial highways and green line
is the Niagara Escarpment)}
\end{figure}

\begin{figure}
\centering
\includegraphics{BE-correlates-cycling-flows_files/figure-latex/residuals-under-5km-1.pdf}
\caption{\label{fig:residuals-under-5km} Map of Under-predicted Bicycle
Trip Flows Less Than 5 km (Black lines are provincial highways and green
line is the Niagara Escarpment)}
\end{figure}

\hypertarget{sec:conclusion}{%
\section{Conclusion}\label{sec:conclusion}}

The objective of this study was to address the following questions: 1)
\emph{Which attributes at the zones of origin and destination influence
cycling trip flows in Hamilton?}; and 2) \emph{Which type of route best
explains the pattern of travel by bicycle in Hamilton?}. The use of a
spatial interaction model is methodologically more holistic than trip
generation analysis, an approach often used in the cycling literature
(for instance, see Noland, Smart, and Guo 2016), because it considered
attributes at the zones of origin and destination, as well as route
characteristics to estimate cyclist travel. Use of a routing algorithm
like \emph{CycleStreets} also constitutes a novel approach to overcome
the limitation of travel surveys. \emph{CycleStreets} enabled us to
experiment with different types of routes that cyclists may seek out.
The model revealed that shortest-distance \emph{quietest} routes that
allow cyclists to avoid traffic best explain the pattern of travel by
bicycle in Hamilton. In addition, the availability of jobs and different
land uses and destinations at the end of the trip were positive
attractors of bicycle trips. Commercial locations and other destinations
at the zone of origin, as well as topography, had a negative influence
on the number of expected bicycle trips. Other findings include that the
misspecification in the analysis of bicycle flows is evident in the form
of network autocorrelation - this has been known for other types of
flows, but as far as we know, has never been reported in the cycling
literature. By testing for network autocorrelation, we are confident in
the final model, which not only accounts for various pull-push factors
and cost measures, but also indicates that the model sufficiently
describes the pattern observed. Finally, analysis of the model residuals
to identify under- and over-estimated bicycle flows was also suggestive
in terms of other information about potential cycling routes.

Broach et al.~(2012) noted that in my cases the conventional travel
demand model does not address cycling well for several reasons. Cycling
is often combined with walking since they are both active modes and it
is often excluded after the second step of the travel demand model,
meaning that route choice and network assignment are not accounted for.
Chen at al.~(2018) touch upon this as well by suggesting that data about
route choice is needed to overcome these limitations. Common approaches
of including only the shortest path route between origins and
destinations when accounting for cycling in a travel demand model
presents additional limitations because it excludes different built
environment attributes that are known to influence route choice (Broach,
Dill, and Gliebe 2012). Use of a routing algorithm helps to overcome the
dearth of information on actual routes and can account for variability
in route characteristics depending on the availability of data. However,
there are advantages and limitations to using cycle routing algorithms.
The ability to infer distance and time from different routes that a
knowledgeable cyclist would take when modelling bicycle trips using data
from travel surveys is particularly efficient when GPS data are not
available. Thus, cycle routing algorithms can be more practical for
transportation planners because they are less demanding and expensive
than collecting route data in travel surveys or creating their own
network dataset. A limitation, on the other hand, is the inability to
capture the variety of routes that cyclists actually take. GPS data,
when available, is more suited to capturing variations between dominant
and shortest path routes (Lu, Scott, and Dalumpines 2018). However,
despite some limitations, we offer that the approach outlined in this
research can be replicated in other cities covered by
\emph{OpenStreetMap} and that strengthening publicly available data in
this portal could be useful to measure the influence of route
characteristics on travel by bicycle between different origins and
destinations.

The approach adopted in this research also presents future opportunities
to systematically investigate the built environment along the inferred
routes. For instance, shortest-path \emph{quietest} routes may have
attributes that promote travel by bicycle, such as infrastructure or a
large proportion of residential streets, which leads to more cycling
than expected from the model. To test this assumption, environmental
audits were conducted along \emph{quietest} routes for a selection of
origin-destination pairs that were under-predicted in order to document
the presence or absence of features that may influence cycling (see
Moniruzzaman and Páez 2012). The documentation of built environment
attributes would contribute to our understanding of what cyclists
experience as they travel through the city of Hamilton as well as
validate whether the inferred routes match where cyclists do indeed
travel. This is the topic of another forthcoming paper {[}Desjardins et
al.~2020b submitted for publication{]}.

\hypertarget{sec:acknowledgments}{%
\section{Acknowledgments}\label{sec:acknowledgments}}

The authors wish to express their gratitude to Yongwan Chun and Roberto
Patuelli for sharing their \texttt{R} code for the Jacqmin-Gadda's \(T\)
test. In addition, the following \texttt{R} packages were used in the
course of this investigation and the authors wish to acknowledge their
developers: \texttt{cyclestreets} (Lovelace and Lucas-Smith 2018),
\texttt{ggthemes} (Arnold 2019), \texttt{kableExtra} (Zhu 2019),
\texttt{knitr}(Xie 2014, 2015), \texttt{rticles} (Allaire et al. 2020),
\texttt{sf} (Pebesma 2018), \texttt{spdep} (Bivand, Pebesma, and
Gomez-Rubio 2013), \texttt{tidyverse} (Wickham et al. 2019),
\texttt{units} (Pebesma, Mailund, and Hiebert 2016), and
\texttt{zeligverse} (Gandrud 2017).

\hypertarget{references}{%
\section*{References}\label{references}}
\addcontentsline{toc}{section}{References}

\textbackslash end\{document\}

\hypertarget{refs}{}
\leavevmode\hypertarget{ref-Adam2020}{}%
Adam, Lukas, Tim Jones, and Marco te Brömmelstroet. 2020. ``Planning for
Cycling in the Dispersed City: Establishing a Hierarchy of Effectiveness
of Municipal Cycling Policies.'' \emph{Transportation} 47 (2): 503--27.
\url{https://doi.org/10.1007/s11116-018-9878-3}.

\leavevmode\hypertarget{ref-Allaire2020}{}%
Allaire, JJ, Yihui Xie, R Foundation, Hadley Wickham, Journal of
Statistical Software, Ramnath Vaidyanathan, Association for Computing
Machinery, et al. 2020. \emph{Rticles: Article Formats for R Markdown}.
\url{https://CRAN.R-project.org/package=rticles}.

\leavevmode\hypertarget{ref-Arnold2019}{}%
Arnold, Jeffrey B. 2019. \emph{Ggthemes: Extra Themes, Scales and Geoms
for 'Ggplot2'}. \url{https://CRAN.R-project.org/package=ggthemes}.

\leavevmode\hypertarget{ref-Assuncao2019}{}%
Assunçao-Denis, Marie-Ève, and Ray Tomalty. 2019. ``Increasing Cycling
for Transportation in Canadian Communities: Understanding What Works.''
\emph{Transportation Research Part A: Policy and Practice} 123 (May):
288--304. \url{https://doi.org/10.1016/j.tra.2018.11.010}.

\leavevmode\hypertarget{ref-Avila2018}{}%
Avila-Palencia, I, L Int Panis, E Dons, and M et al. Gaupp-Berghausen.
2018. ``The Effects of Transport Mode Use on Self-Perceived Health,
Mental Health, and Social Contact Measures: A Cross-Sectional and
Longitudinal Study.'' Journal Article. \emph{Environment International}
120: 199--206.
\url{https://doi.org/https://doi.org/10.1016/j.envint.2018.08.002}.

\leavevmode\hypertarget{ref-Bivand2013}{}%
Bivand, Roger S., Edzer Pebesma, and Virgilio Gomez-Rubio. 2013.
\emph{Applied Spatial Data Analysis with R, Second Edition}. Springer,
NY. \url{http://www.asdar-book.org/}.

\leavevmode\hypertarget{ref-Branion2019}{}%
Branion-Calles, Michael, Trisalyn Nelson, Daniel Fuller, Lise Gauvin,
and Meghan Winters. 2019. ``Associations Between Individual
Characteristics, Availability of Bicycle Infrastructure, and City-Wide
Safety Perceptions of Bicycling: A Cross-Sectional Survey of Bicyclists
in 6 Canadian and U.S. Cities.'' \emph{Transportation Research Part A:
Policy and Practice} 123 (May): 229--39.
\url{https://doi.org/10.1016/j.tra.2018.10.024}.

\leavevmode\hypertarget{ref-Broach2012}{}%
Broach, Joseph, Jennifer Dill, and John Gliebe. 2012. ``Where Do
Cyclists Ride? A Route Choice Model Developed with Revealed Preference
Gps Data.'' \emph{Transportation Research Part A: Policy and Practice}
46 (10): 1730--40. \url{https://doi.org/10.1016/j.tra.2012.07.005}.

\leavevmode\hypertarget{ref-Buehler2016}{}%
Buehler, Ralph, and Jennifer Dill. 2016. ``Bikeway Networks: A Review of
Effects on Cycling.'' \emph{Transport Reviews} 36 (1): 9--27.
\url{https://doi.org/10.1080/01441647.2015.1069908}.

\leavevmode\hypertarget{ref-Buehler2012}{}%
Buehler, Ralph, and John Pucher. 2012. ``Cycling to Work in 90 Large
American Cities: New Evidence on the Role of Bike Paths and Lanes.''
\emph{Transportation} 39 (2): 409--32.
\url{https://doi.org/10.1007/s11116-011-9355-8}.

\leavevmode\hypertarget{ref-Celis2017}{}%
Celis-Morales, CA, DM Lyall, and P et al. Welsh. 2017. ``Association
Between Active Commuting and Incident Cardiovascular Disease, Cancer,
and Mortality: Prospective Cohort Study.'' Journal Article. \emph{BMJ}
357: j1456. \url{https://doi.org/https://doi.org/10.1136/bmj.j1456}.

\leavevmode\hypertarget{ref-Cervero2019}{}%
Cervero, Robert, Steve Denman, and Ying Jin. 2019. ``Network Design,
Built and Natural Environments, and Bicycle Commuting: Evidence from
British Cities and Towns.'' \emph{Transport Policy} 74 (February):
153--64. \url{https://doi.org/10.1016/j.tranpol.2018.09.007}.

\leavevmode\hypertarget{ref-Chen2018}{}%
Chen, Peng, Qing Shen, and Suzanne Childress. 2018. ``A Gps Data-Based
Analysis of Built Environment Influences on Bicyclist Route
Preferences.'' \emph{International Journal of Sustainable
Transportation} 12 (3): 218--31.
\url{https://doi.org/10.1080/15568318.2017.1349222}.

\leavevmode\hypertarget{ref-Chun2008}{}%
Chun, Yongwan. 2008. ``Modeling Network Autocorrelation Within Migration
Flows by Eigenvector Spatial Filtering.'' \emph{Journal of Geographical
Systems} 10 (4): 317--44.
\url{https://doi.org/10.1007/s10109-008-0068-2}.

\leavevmode\hypertarget{ref-Calgary2011}{}%
City of Calgary. 2011. ``Cycling Strategy.'' 2011.
\url{https://www.calgary.ca/Transportation/TP/Documents/cycling/Cycling-Strategy/2011-cycling-strategy-presentation.pdf}.

\leavevmode\hypertarget{ref-Tmp2018}{}%
City of Hamilton. 2018a. ``City of Hamilton Transportation Master Plan
Review and Update.'' 2018.
\url{https://www.hamilton.ca/sites/default/files/media/browser/2018-10-24/tmp-review-update-final-report-oct2018.pdf}.

\leavevmode\hypertarget{ref-Cmp2018}{}%
---------. 2018b. ``Cycling Master Plan Review and Update.'' 2018.
\url{https://www.hamilton.ca/sites/default/files/media/browser/2018-06-06/draft-tmp-backgroundreport-cyclingmp-11-1.pdf}.

\leavevmode\hypertarget{ref-Cmp2009}{}%
---------. 2018c. ``Shifting Gears 2009: Hamilton's Cycling Master Plan
Review and Update.'' 2018.
\url{https://www.hamilton.ca/sites/default/files/media/browser/2014-12-17/cycling-master-plan-chapters-1-2-3.pdf}.

\leavevmode\hypertarget{ref-Montreal2017}{}%
City of Montreal. 2017. ``Montreal, City of Cyclists; Cycling Master
Plan: Safety, Efficiency, Audacity.'' 2017.
\url{https://ville.montreal.qc.ca/pls/portal/docs/page/transports_fr/media/documents/plan_cadre_velo_ang_final_lr.pdf}.

\leavevmode\hypertarget{ref-Vancouver2012}{}%
City of Vancouver. 2012. ``Transportation 2040: Moving Forward.'' 2012.
\url{https://vancouver.ca/files/cov/transportation-2040-plan.pdf}.

\leavevmode\hypertarget{ref-ColeHunter2015}{}%
Cole-Hunter, T, D Donaire-Gonzalez, A Curto, A Ambros, A Valentin, J
Garcia-Aymerich, D Martínez, et al. 2015. ``Objective Correlates and
Determinants of Bicycle Commuting Propensity in an Urban Environment.''
\emph{Transportation Research Part D: Transport and Environment} 40
(October): 132--43. \url{https://doi.org/10.1016/j.trd.2015.07.004}.

\leavevmode\hypertarget{ref-Dmg2014tts}{}%
Data Management Group. 2014. ``2011 Tts: Design and Conduct of the
Survey.'' 2014.
\url{http://dmg.utoronto.ca/pdf/tts/2011/conduct2011.pdf}.

\leavevmode\hypertarget{ref-Dmg2018tts}{}%
---------. 2018. ``2016 Tts: Design and Conduct of the Survey.'' 2018.
\url{http://dmg.utoronto.ca/pdf/tts/2016/2016TTS_Conduct.pdf}.

\leavevmode\hypertarget{ref-deNazelle2011}{}%
De Nazelle, Audrey, Mark J Nieuwenhuijsen, Josep M Antó, Michael Brauer,
David Briggs, Charlotte Braun-Fahrlander, Nick Cavill, et al. 2011.
``Improving Health Through Policies That Promote Active Travel: A Review
of Evidence to Support Integrated Health Impact Assessment.''
\emph{Environment International} 37 (4): 766--77.

\leavevmode\hypertarget{ref-Dill2003}{}%
Dill, J, and T Carr. 2003. ``Bicycle Commuting and Facilities in Major
U.s. Cities: If You Build Them, Commuters Will Use Them.'' Journal
Article. \emph{Transportation Research Record: Journal of the
Transportation Research Board} 1828.

\leavevmode\hypertarget{ref-Dill2009}{}%
Dill, Jennifer. 2009. ``Bicycling for Transportation and Health: The
Role of Infrastructure.'' \emph{Journal of Public Health Policy} 30
(S1): S95--S110. \url{https://doi.org/10.1057/jphp.2008.56}.

\leavevmode\hypertarget{ref-deDios2011Modelling}{}%
Dios Ortúzar, Juan de, and Luis G Willumsen. 2011. \emph{Modelling
Transport}. 4th ed. John wiley \& sons.

\leavevmode\hypertarget{ref-elAssi2017effects}{}%
El-Assi, Wafic, Mohamed Salah Mahmoud, and Khandker Nurul Habib. 2017.
``Effects of Built Environment and Weather on Bike Sharing Demand: A
Station Level Analysis of Commercial Bike Sharing in Toronto.''
\emph{Transportation} 44 (3): 589--613.

\leavevmode\hypertarget{ref-Gandrud2017}{}%
Gandrud, Christopher. 2017. \emph{Zeligverse: Easily Install and Load
Stable Zelig Packages}.
\url{https://CRAN.R-project.org/package=zeligverse}.

\leavevmode\hypertarget{ref-Griffith2011}{}%
Griffith, Daniel A. 2011. ``Visualizing Analytical Spatial
Autocorrelation Components Latent in Spatial Interaction Data: An
Eigenvector Spatial Filter Approach.'' \emph{Computers, Environment and
Urban Systems} 35 (2): 140--49.
\url{https://doi.org/10.1016/j.compenvurbsys.2010.08.003}.

\leavevmode\hypertarget{ref-handyPromotingCyclingTransport2014}{}%
Handy, Susan, Bert van Wee, and Maarten Kroesen. 2014. ``Promoting
Cycling for Transport: Research Needs and Challenges.'' \emph{Transport
Reviews} 34: 4--24. \url{https://doi.org/10.1080/01441647.2013.860204}.

\leavevmode\hypertarget{ref-Heesch2015}{}%
Heesch, Kristiann C., Billie Giles-Corti, and Gavin Turrell. 2015.
``Cycling for Transport and Recreation: Associations with the
Socio-Economic, Natural and Built Environment.'' \emph{Health \& Place}
36 (November): 152--61.
\url{https://doi.org/10.1016/j.healthplace.2015.10.004}.

\leavevmode\hypertarget{ref-heinenCommutingBicycleOverview2010}{}%
Heinen, Eva, Bert van Wee, and Kees Maat. 2010. ``Commuting by Bicycle:
An Overview of the Literature.'' \emph{Transport Reviews} 30: 59--96.
\url{https://doi.org/10.1080/01441647.2013.860204}.

\leavevmode\hypertarget{ref-Jacqmin1997}{}%
Jacqmin-Gadda, Hélène, Daniel Commenges, Chakib Nejjari, and
Jean-François Dartigues. 1997. ``Tests of Geographical Correlation with
Adjustment for Explanatory Variables: An Application to Dyspnoea in the
Elderly.'' \emph{Statistics in Medicine} 16 (11): 1283--97.

\leavevmode\hypertarget{ref-Le2018}{}%
Le, Huyen T. K., Ralph Buehler, and Steve Hankey. 2018. ``Correlates of
the Built Environment and Active Travel: Evidence from 20 Us
Metropolitan Areas.'' \emph{Environmental Health Perspectives} 126 (7):
077011. \url{https://doi.org/10.1289/EHP3389}.

\leavevmode\hypertarget{ref-liuWhatMakesGood2020}{}%
Liu, George, Samuel Nello-Deakin, Marco Brommelstroet te, and Yuki
Yamamoto. 2020. ``What Makes a Good Cargo Bike Route? Perspectives from
Users and Planners.'' \emph{American Journal of Economics and Sociology}
73: 941--65. \url{https://doi.org/10.1111/ajes.12332}.

\leavevmode\hypertarget{ref-Lovelace2018}{}%
Lovelace, Robin, and Martin Lucas-Smith. 2018. \emph{Cyclestreets: Cycle
Routing and Data for Cycling Advocacy}.
\url{https://CRAN.R-project.org/package=cyclestreets}.

\leavevmode\hypertarget{ref-Lu2018understanding}{}%
Lu, Wei, Darren M Scott, and Ron Dalumpines. 2018. ``Understanding Bike
Share Cyclist Route Choice Using Gps Data: Comparing Dominant Routes and
Shortest Paths.'' \emph{Journal of Transport Geography} 71: 172--81.

\leavevmode\hypertarget{ref-McArthur2019}{}%
McArthur, David Philip, and Jinhyun Hong. 2019. ``Visualising Where
Commuting Cyclists Travel Using Crowdsourced Data.'' \emph{Journal of
Transport Geography} 74 (January): 233--41.
\url{https://doi.org/10.1016/j.jtrangeo.2018.11.018}.

\leavevmode\hypertarget{ref-Mertens2017}{}%
Mertens, Lieze, Sofie Compernolle, Benedicte Deforche, Joreintje D.
Mackenbach, Jeroen Lakerveld, Johannes Brug, Célina Roda, et al. 2017.
``Built Environmental Correlates of Cycling for Transport Across
Europe.'' \emph{Health \& Place} 44 (March): 35--42.
\url{https://doi.org/10.1016/j.healthplace.2017.01.007}.

\leavevmode\hypertarget{ref-Metulini2018}{}%
Metulini, Rodolfo, Roberto Patuelli, and Daniel Griffith. 2018. ``A
Spatial-Filtering Zero-Inflated Approach to the Estimation of the
Gravity Model of Trade.'' \emph{Econometrics} 6 (1): 9.
\url{https://doi.org/10.3390/econometrics6010009}.

\leavevmode\hypertarget{ref-Mitra2016}{}%
Mitra, R, N Smith Lea, I Cantello, and G Hanson. 2016. ``Cycling
Behaviour and Potential in the Greater Toronto and Hamilton Area.''
2016.
\url{http://transformlab.ryerson.ca/wp-content/uploads/2016/10/Cycling-potential-in-GTHA-final-report-2016.pdf}.

\leavevmode\hypertarget{ref-Moniruzzaman2012}{}%
Moniruzzaman, M.d., and A Páez. 2012. ``A Model-Based Approach to Select
Case Sites for Walkability Audits.'' Journal Article. \emph{Health \&
Place} 18 (6): 1323--34.
\url{https://doi.org/10.1016/j.healthplace.2012.09.013}.

\leavevmode\hypertarget{ref-Moniruzzaman2016}{}%
---------. 2016. ``An Investigation of the Attributes of Walkable
Environments from the Perspective of Seniors in Montreal.'' Journal
Article. \emph{Journal of Transport Geography} 51: 85--96.
\url{https://doi.org/http://dx.doi.org/10.1016/j.jtrangeo.2015.12.001}.

\leavevmode\hypertarget{ref-Moudon2003walking}{}%
Moudon, Anne Vernez, and Chanam Lee. 2003. ``Walking and Bicycling: An
Evaluation of Environmental Audit Instruments.'' \emph{American Journal
of Health Promotion} 18 (1): 21--37.

\leavevmode\hypertarget{ref-moudonCyclingBuiltEnvironment2005a}{}%
Moudon, Anne Vernez, Chanam Lee, Allen D. Cheadle, Cheza W. Collier,
Donna Johnson, Thomas L. Schmid, and Robert D. Weather. 2005. ``Cycling
and the Built Environment, a Us Perspective.'' \emph{Transportation
Research Part D: Transport and Environment} 10 (3): 245--61.

\leavevmode\hypertarget{ref-Nielsen2018}{}%
Nielsen, Thomas Alexander Sick, and Hans Skov-Petersen. 2018.
``Bikeability -- Urban Structures Supporting Cycling. Effects of Local,
Urban and Regional Scale Urban Form Factors on Cycling from Home and
Workplace Locations in Denmark.'' \emph{Journal of Transport Geography}
69 (May): 36--44. \url{https://doi.org/10.1016/j.jtrangeo.2018.04.015}.

\leavevmode\hypertarget{ref-nolandBikeshareTripGeneration2016}{}%
Noland, Robert B., Michael J. Smart, and Ziye Guo. 2016. ``Bikeshare
Trip Generation in New York City.'' \emph{Transportation Research Part
A: Policy and Practice} 94: 164--81.
\url{https://doi.org/10.1016/j.tra.2016.08.030}.

\leavevmode\hypertarget{ref-Nordengen2019}{}%
Nordengen, Ruther, Riiser, Andersen, and Solbraa. 2019. ``Correlates of
Commuter Cycling in Three Norwegian Counties.'' \emph{International
Journal of Environmental Research and Public Health} 16 (22): 4372.
\url{https://doi.org/10.3390/ijerph16224372}.

\leavevmode\hypertarget{ref-Oja2011}{}%
Oja, P., S. Titze, A. Bauman, B. de Geus, P. Krenn, B. Reger-Nash, and
T. Kohlberger. 2011. ``Health Benefits of Cycling: A Systematic
Review.'' \emph{Scandinavian Journal of Medicine \& Science in Sports}
21 (4): 496--509.
\url{https://doi.org/10.1111/j.1600-0838.2011.01299.x}.

\leavevmode\hypertarget{ref-paezEnjoymentCommuteComparison2010}{}%
Páez, Antonio, and Kate Whalen. 2010. ``Enjoyment of Commute: A
Comparison of Different Transportation Modes.'' \emph{Transportation
Research Part A: Policy and Practice} 44 (7): 537--49.
\url{https://doi.org/10.1016/j.tra.2010.04.003}.

\leavevmode\hypertarget{ref-Pebesma2018}{}%
Pebesma, Edzer. 2018. ``Simple Features for R: Standardized Support for
Spatial Vector Data.'' \emph{The R Journal} 10 (1): 439--46.
\url{https://doi.org/10.32614/RJ-2018-009}.

\leavevmode\hypertarget{ref-Pebesma2016}{}%
Pebesma, Edzer, Thomas Mailund, and James Hiebert. 2016. ``Measurement
Units in R.'' \emph{R Journal} 8 (2): 486--94.
\url{https://doi.org/10.32614/RJ-2016-061}.

\leavevmode\hypertarget{ref-Pritchard2018}{}%
Pritchard, Ray. 2018. ``Revealed Preference Methods for Studying Bicycle
Route Choice - a Systematic Review.'' \emph{International Journal of
Environmental Research and Public Health} 15 (3): 470.
\url{https://doi.org/10.3390/ijerph15030470}.

\leavevmode\hypertarget{ref-Pritchard2019}{}%
Pritchard, Ray, Dominik Bucher, and Yngve Frøyen. 2019. ``Does New
Bicycle Infrastructure Result in New or Rerouted Bicyclists? A
Longitudinal Gps Study in Oslo.'' \emph{Journal of Transport Geography}
77 (May): 113--25. \url{https://doi.org/10.1016/j.jtrangeo.2019.05.005}.

\leavevmode\hypertarget{ref-Pucher2006}{}%
Pucher, John, and Ralph Buehler. 2006. ``Why Canadians Cycle More Than
Americans: A Comparative Analysis of Bicycling Trends and Policies.''
\emph{Transport Policy} 13 (3): 265--79.
\url{https://doi.org/10.1016/j.tranpol.2005.11.001}.

\leavevmode\hypertarget{ref-Sallis2013}{}%
Sallis, James F., Terry L. Conway, Lianne I. Dillon, Lawrence D. Frank,
Marc A. Adams, Kelli L. Cain, and Brian E. Saelens. 2013.
``Environmental and Demographic Correlates of Bicycling.''
\emph{Preventive Medicine} 57 (5): 456--60.
\url{https://doi.org/10.1016/j.ypmed.2013.06.014}.

\leavevmode\hypertarget{ref-Sarjala2019}{}%
Sarjala, Satu. 2019. ``Built Environment Determinants of Pedestrians'
and Bicyclists' Route Choices on Commute Trips: Applying a New
Grid-Based Method for Measuring the Built Environment Along the Route.''
\emph{Journal of Transport Geography} 78 (June): 56--69.
\url{https://doi.org/10.1016/j.jtrangeo.2019.05.004}.

\leavevmode\hypertarget{ref-Schneider2015}{}%
Schneider, Robert J., and Joseph Stefanich. 2015. ``Neighborhood
Characteristics That Support Bicycle Commuting: Analysis of the Top 100
U.s. Census Tracts.'' \emph{Transportation Research Record: Journal of
the Transportation Research Board} 2520: 41--51.
\url{https://doi.org/DOI:\%2010.3141/2520-06}.

\leavevmode\hypertarget{ref-Scott2019factors}{}%
Scott, Darren M, and Celenna Ciuro. 2019. ``What Factors Influence Bike
Share Ridership? An Investigation of Hamilton, Ontario's Bike Share
Hubs.'' \emph{Travel Behaviour and Society} 16: 50--58.

\leavevmode\hypertarget{ref-SkovPetersen2018}{}%
Skov-Petersen, Hans, Bernhard Barkow, Thomas Lundhede, and Jette Bredahl
Jacobsen. 2018. ``How Do Cyclists Make Their Way? - a Gps-Based Revealed
Preference Study in Copenhagen.'' \emph{International Journal of
Geographical Information Science} 32 (7): 1469--84.
\url{https://doi.org/10.1080/13658816.2018.1436713}.

\leavevmode\hypertarget{ref-Statscan2017}{}%
Statistics Canada. 2017. ``Journey to Work: Key Results from the 2016
Census.'' 2017.
\url{https://www150.statcan.gc.ca/n1/en/daily-quotidien/171129/dq171129c-eng.pdf?st=eVPg5Nih}.

\leavevmode\hypertarget{ref-Verlinden2019}{}%
Verlinden, Y and Manaugh, K and Savan, B and Smith Lea, N and Tomalty, R
and Winters, M. 2019. ``Increasing Cycling in Canada: A Guide to What
Works.'' 2019.
\url{https://www.tcat.ca/wp-content/uploads/2019/09/Increasing-Cycling-in-Canada-A-Guide-to-What-Works-2019-09-25.pdf}.

\leavevmode\hypertarget{ref-Wickham2019}{}%
Wickham, Hadley, Mara Averick, Jennifer Bryan, Winston Chang, Lucy
D'Agostino McGowan, Romain François, Garrett Grolemund, et al. 2019.
``Welcome to the tidyverse.'' \emph{Journal of Open Source Software} 4
(43): 1686. \url{https://doi.org/10.21105/joss.01686}.

\leavevmode\hypertarget{ref-Winters2010}{}%
Winters, Meghan, Michael Brauer, Eleanor M. Setton, and Kay Teschke.
2010. ``Built Environment Influences on Healthy Transportation Choices:
Bicycling Versus Driving.'' \emph{Journal of Urban Health} 87 (6):
969--93. \url{https://doi.org/10.1007/s11524-010-9509-6}.

\leavevmode\hypertarget{ref-Xie2014}{}%
Xie, Yihui. 2014. ``Knitr: A Comprehensive Tool for Reproducible
Research in R.'' In \emph{Implementing Reproducible Computational
Research}, edited by Victoria Stodden, Friedrich Leisch, and Roger D.
Peng. Chapman; Hall/CRC.
\url{http://www.crcpress.com/product/isbn/9781466561595}.

\leavevmode\hypertarget{ref-Xie2015}{}%
---------. 2015. \emph{Dynamic Documents with R and Knitr}. 2nd ed. Boca
Raton, Florida: Chapman; Hall/CRC. \url{https://yihui.org/knitr/}.

\leavevmode\hypertarget{ref-yangCyclingfriendlyCityUpdated2019}{}%
Yang, Yiyang, Xueying Wu, Peiling Zhao, Gou Zhonghua, and Yi Lu. 2019.
``Towards a Cycling-Friendly City: An Updated Review of the Associations
Between Built Environment and Cycling Behaviors (2007--2017).''
\emph{Journal of Transport \& Health} 14: 100613.
\url{https://doi.org/10.1016/j.jth.2019.100613}.

\leavevmode\hypertarget{ref-Zahabi2016}{}%
Zahabi, Seyed Amir H., Annie Chang, Luis F. Miranda-Moreno, and Zachary
Patterson. 2016. ``Exploring the Link Between the Neighborhood
Typologies, Bicycle Infrastructure and Commuting Cycling over Time and
the Potential Impact on Commuter Ghg Emissions.'' \emph{Transportation
Research Part D: Transport and Environment} 47 (August): 89--103.
\url{https://doi.org/10.1016/j.trd.2016.05.008}.

\leavevmode\hypertarget{ref-Zhao2014}{}%
Zhao, Pengjun. 2014. ``The Impact of the Built Environment on Bicycle
Commuting: Evidence from Beijing.'' \emph{Urban Studies} 51 (5):
1019--37. \url{https://doi.org/10.1177/0042098013494423}.

\leavevmode\hypertarget{ref-Zhu2019}{}%
Zhu, Hao. 2019. \emph{KableExtra: Construct Complex Table with 'Kable'
and Pipe Syntax}. \url{https://CRAN.R-project.org/package=kableExtra}.

\bibliographystyle{spphys}
\bibliography{References.bib}

\end{document}
